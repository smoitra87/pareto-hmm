\documentclass{article}
\usepackage{graphicx}
\usepackage{amsmath,amssymb,amsthm}
%\usepackage[square]{natbib}
\usepackage{color}
\usepackage{algpseudocode}
\usepackage{algorithm}
\usepackage{multirow}
\usepackage{color}
\usepackage[usenames,dvipsnames,svgnames,table]{xcolor}


\DeclareMathOperator*{\argmax}{arg\,max}
\DeclareMathOperator*{\argmin}{arg\,min}

\begin{document}

\section{Experiments}
In this section we evaluate the performance of our algorithm on simulated as well as experimental data. 

\subsection{Simulation Experiments}
We run a variety of different simulation experiments to check the performance of our algorithm on 

\subsubsection{Sim Experiment 1}
\label{sim:toy}
We demonstrate how our algorithm performs on simulated data. The data was prepared by considering a protein of length 12. The sequence space was limited to an alphabet consisting of two arbitrary characters $\alpha,\beta$. The observation belong to a discrete space Helix, Beta and Loop which represented by H,B and L. The weights of the CMRF have weights $\phi,\psi$ as defined in Tables~\ref{tab:psi1} and~\ref{tab:psi2}. We define two states State A and State B that corresponding to competing objectives for multistate design. State A is a Helix-Loop-Helix strand whereas State B is Sheet-Loop-Sheet strand. Figure~\ref{fig:simstates} shows a cartoon diagrams for our design problem. State A and State B have \texttt{HHHHLLLLHHHH} and \texttt{BBBBLLLLBBBB} secondary structure features respectively. These are the observations for the CMRF.
\\
\\
Using the weight matrices we defined in Tables~\ref{tab:psi1} and~\ref{tab:psi2} we run CMRF-Decode to get $X_A$ which satisfies $X_A = \argmin_X E_A(X)$. We obtain $X_B$ similarly. Not surprisingly $X_A$ is $\alpha\alpha\alpha\alpha\alpha\alpha\alpha\alpha\alpha\alpha\alpha\alpha$ 
and $X_B$ is $\beta\beta\beta\beta\beta\beta\beta\beta\beta\beta\beta\beta$. This is because the weights in the table are biased towards $\alpha$ emitting \texttt{H} and $\beta$ emitting \texttt{B}. Thus according to our model $X_A$ is the sequence that is most stable under state A, whereas $X_B$ is the sequence is most stable under state B.
\\
\\
We then run the \texttt{Pareto-Frontier} algorithm using the same weights and observations for state A and state B to enumerate the entire pareto frontier. This pareto frontier contains all the best sequences that any convex combination of $E_A(X)$ and $E_B(X)$ would give rise to. We number the points to show the path taken by the pareto frontier algorithm. We also calculate the energies of all the $2^{12}$ sequences possible under this restricted sequence model and plot the points. The \texttt{Pareto-Frontier} algorithm was able to find the pareto frontier in \textcolor{red}{1 step} steps. See Figure~\ref{fig:pareto-sim} for the trace taken by the algorithm.  We also compare the running times as opposed to brute force and MCMC and compare the running times. 

\begin{table}
\begin{center}
\begin{tabular}{cc|c|c|c|}
\cline{3-5}
& & \multicolumn{3}{c|}{Observations($\mathbf{O_i}$)} \\ \cline{3-5}
& & Helix(H) & Beta(B) & Loop(L) \\ \cline{1-5}
\multicolumn{1}{|c|}{\multirow{2}{*}{Sequence($X_i$)}} &
\multicolumn{1}{|c|}{$\alpha$} & 0.6 & 0.2 & 0.2      \\ \cline{2-5}
\multicolumn{1}{|c|}{}                        &
\multicolumn{1}{|c|}{$\beta$} & 0.2 & 0.6 & 0.2      \\ \cline{1-5}
\end{tabular}
\end{center}
\caption{$\psi(X,\mathbf{O})$ contains the weights for edge potentials of sequence and features together }
\label{tab:psi1}
\end{table}


\begin{table}
\begin{center}
\begin{tabular}{cc|c|c|}
\cline{3-4}
& & \multicolumn{2}{c|}{Sequence($X_{i+1}$)} \\ \cline{3-4}
& & $\alpha$ & $\beta$  \\ \cline{1-4}
\multicolumn{1}{|c|}{\multirow{2}{*}{Sequence($X_i$)}} &
\multicolumn{1}{|c|}{$\alpha$} & 0.7 & 0.3      \\ \cline{2-4}
\multicolumn{1}{|c|}{}                        &
\multicolumn{1}{|c|}{$\beta$} & 0.3 & 0.7      \\ \cline{1-4}
\end{tabular}
\end{center}
\caption{$\psi(X_i,X_{i+1})$ contains the weights for the edge potentials of adjacent amino acids}
\label{tab:psi2}
\end{table}

\begin{table}
\begin{center}
\begin{tabular}{cc|c|c|}
\cline{3-4}
& & \multicolumn{2}{c|}{Sequence($X_{i+1}$)} \\ \cline{3-4}
& & $\alpha$ & $\beta$  \\ \cline{1-4}
\multicolumn{1}{|c|}{\multirow{2}{*}{Sequence($X_i$)}} &
\multicolumn{1}{|c|}{$\alpha$} & 0.7 & 0.3      \\ \cline{2-4}
\multicolumn{1}{|c|}{}                        &
\multicolumn{1}{|c|}{$\beta$} & 0.3 & 0.7      \\ \cline{1-4}
\end{tabular}
\end{center}
\caption{Time and steps taken on simulated data}
\end{table}

\begin{figure}[h!]
    \centering
    \includegraphics[width=0.9\textwidth]{pics/sim1.pdf}
    \caption{Simulation Experiment 1: The emission and transition weights are as defined in Tables~\ref{tab:psi1} and~\ref{tab:psi2}. The Pareto-HMM finds the optimal pareto frontier for a twelve length sequence using the alphabet \texttt\{a,b\} and the feature space \texttt\{H,L,B\} in two steps of the Convex hull algorithm. The magenta stars corresponds to the optimal sequences. The magenta line corresponds to the lower left convex hull of the sequence energy space. The blue stars correspond to non-optimal dominated sequences.  The Pareto-Frontier algoerithm provably finds the frontier quickly instead of having to enumerate,calculate and sort the energies of a  $2^{12}$ sequence space.}
    \label{fig:pareto-sim}
\end{figure}

\pagebreak

\subsubsection{Simulation Experiment 2}
\label{sim:randprobs}
In this example we look at a slightly more complicated energy frontier. To generate this energy surface we resorted to generating random emission and transition probabilities till we found a sufficiently complicated energy frontier but still having alphabet \texttt{\{a,b\}} and feature space \texttt{\{H,L,B\}}. The probabilities for this simulation experiment are shown in tables~\ref{tab:psi3} and~\ref{tab:psi4}. 

\begin{table}
\begin{center}
\begin{tabular}{cc|c|c|c|}
\cline{3-5}
& & \multicolumn{3}{c|}{Observations($\mathbf{O_i}$)} \\ \cline{3-5}
& & Helix(H) & Beta(B) & Loop(L) \\ \cline{1-5}
\multicolumn{1}{|c|}{\multirow{2}{*}{Sequence($X_i$)}} &
\multicolumn{1}{|c|}{$\alpha$} & 0.334 & 0.272 & 0.394      \\ \cline{2-5}
\multicolumn{1}{|c|}{}                        &
\multicolumn{1}{|c|}{$\beta$} & 0.477 & 0.093 & 0.430      \\ \cline{1-5}
\end{tabular}
\end{center}
\caption{$\psi(X,\mathbf{O})$ contains the weights for edge potentials of sequence and features together }
\label{tab:psi3}
\end{table}


\begin{table}
\begin{center}
\begin{tabular}{cc|c|c|}
\cline{3-4}
& & \multicolumn{2}{c|}{Sequence($X_{i+1}$)} \\ \cline{3-4}
& & $\alpha$ & $\beta$  \\ \cline{1-4}
\multicolumn{1}{|c|}{\multirow{2}{*}{Sequence($X_i$)}} &
\multicolumn{1}{|c|}{$\alpha$} & 0.483 & 0.517      \\ \cline{2-4}
\multicolumn{1}{|c|}{}                        &
\multicolumn{1}{|c|}{$\beta$} & 0.589 & 0.411      \\ \cline{1-4}
\end{tabular}
\end{center}
\caption{$\psi(X_i,X_{i+1})$ contains the weights for the edge potentials of adjacent amino acids}
\label{tab:psi4}
\end{table}


\begin{figure}[h!]
    \centering
    \includegraphics[width=0.9\textwidth]{pics/sim2_all.pdf}
    \caption{Simulation Experiment 2: Pareto Frontier algorithm is applied to a more complicated energy surface. The algorithm finds the pareto frontier denoted by the dashed magenta in 5 steps, which is significantly faster than brute force enumeration.  }
    \label{fig:sim2_all}
\end{figure}

\begin{figure}[h!]
    \centering
    \includegraphics[width=0.9\textwidth]{pics/sim2_frontier.pdf}
    \caption{Simulation Experiment 2(Frontier): This shows the results from simulation 2. The sequences along the frontier are shown as well }
    \label{fig:sim2_frontier}
\end{figure}

\pagebreak
\clearpage

\subsubsection{Simulation Experiment 3: Random probs tied}
\label{sim:randprobstied}
In this experiment we use a 12 length sequence. We do 2-state design using features \texttt{HHHHLLLLHHHH} and \texttt{BBBBLLLLBBBB} as before. The sequence space is restricted to a dummy \texttt{\{a,b\}}  We randomize the emission and transition probabilities. The parameters of the emission and the transition are ties together.  We store the frontier, frontier energies, computation time taken and 

\subsubsection{Simulation Experiment 4:Random probs untied}
\label{sim:randprobsuntied}
Same as Simulation~\ref{sim:randprobstied} with the difference that the emission and the transition probabilities are untied. 

\subsubsection{Simulation Experiment 5 : Random feature space tied}
Like~\ref{sim:randprobstied} but the feature space is randomized while the parameters between differnent amino acids are tied together. 
\label{sim:randfeatstied}

\subsubsection{Simulation Experiment 6 :  Random feature space untied}
Like~\ref{sim:randprobsuntied} but the feature space is randomized while the different amino acids are untied. This is the most general of the previous settings. 
\label{sim:randfeatsuntied}


\begin{center}
    \begin{tabular}{| c | c | c | c |}
    \hline
    Experiment & Brute & Pareto \\ \hline
    Sim~\ref{sim:toy} & 0.907 & 0.003 \\ \hline
    Sim~\ref{sim:randprobs} & 0.986 & 0.006 \\ \hline
    Sim~\ref{sim:randprobstied} & 0.912 & 0.004 \\ \hline
    Sim~\ref{sim:randprobsuntied} & 0.903 & 0.0034\\ \hline
    Sim~\ref{sim:randfeatstied} & 0.921 & 0.011 \\ \hline
    Sim~\ref{sim:randfeatsuntied} & 0.910 & 0.006 \\ \hline
    \hline
    \end{tabular}
\end{center}


\subsubsection{Simulation Experiment 7: Vary Seq space }
\label{sim:seqspace}
Increase the sequence space using random untied feature space as in ~\ref{sim:randfeatsuntied}. Will supply different seq space values. 2,4,16,20 are the values that are goint to be used.
\begin{verbatim}
sim3_seqspace_2_0pareto : 0.017 seconds
sim3_seqspace_2_1pareto : 0.005 seconds
sim3_seqspace_2_2pareto : 0.007 seconds
sim3_seqspace_4_0pareto : 0.022 seconds
sim3_seqspace_4_1pareto : 0.021 seconds
sim3_seqspace_4_2pareto : 0.015 seconds
sim3_seqspace_8_0pareto : 0.040 seconds
sim3_seqspace_8_1pareto : 0.069 seconds
sim3_seqspace_8_2pareto : 0.061 seconds
sim3_seqspace_16_0pareto : 0.338 seconds
sim3_seqspace_16_1pareto : 0.213 seconds
sim3_seqspace_16_2pareto : 0.214 seconds
sim3_seqspace_20_0pareto : 0.370 seconds
sim3_seqspace_20_1pareto : 0.463 seconds
sim3_seqspace_20_2pareto : 0.284 seconds
\end{verbatim}


\subsubsection{Simulation Experiment 8: Seq space 20, Vary seq length}
\label{sim:seqlen}
Set the sequence space to be 20 and vary the sequence length from $2^2$  to $2^7$
\begin{verbatim}
sim3_seqlen_4_0pareto : 0.047 seconds
sim3_seqlen_4_1pareto : 0.011 seconds
sim3_seqlen_4_2pareto : 0.029 seconds
sim3_seqlen_16_0pareto : 0.621 seconds
sim3_seqlen_16_1pareto : 0.857 seconds
sim3_seqlen_16_2pareto : 0.499 seconds
sim3_seqlen_32_0pareto : 2.343 seconds
sim3_seqlen_32_1pareto : 2.344 seconds
sim3_seqlen_32_2pareto : 2.409 seconds
sim3_seqlen_64_0pareto : 8.683 seconds
sim3_seqlen_64_1pareto : 10.716 seconds
sim3_seqlen_64_2pareto : 11.068 seconds
sim3_seqlen_128_0pareto : 39.451 seconds
sim3_seqlen_128_1pareto : 40.050 seconds
sim3_seqlen_128_2pareto : 33.411 seconds
\end{verbatim}
%

\subsubsection{Simulation Experiment 9 : Jointly vary seq space and length}
\label{sim:seqspacelen}
Jointly vary the sequence space and sequence length, 
\begin{verbatim}
sim3_seqspacelen_4_2_0pareto : 0.001 seconds
sim3_seqspacelen_4_2_1pareto : 0.001 seconds
sim3_seqspacelen_4_4_0pareto : 0.000 seconds
sim3_seqspacelen_4_4_1pareto : 0.004 seconds
sim3_seqspacelen_4_8_0pareto : 0.010 seconds
sim3_seqspacelen_4_8_1pareto : 0.003 seconds
sim3_seqspacelen_4_16_0pareto : 0.032 seconds
sim3_seqspacelen_4_16_1pareto : 0.014 seconds
sim3_seqspacelen_16_2_0pareto : 0.019 seconds
sim3_seqspacelen_16_2_1pareto : 0.017 seconds
sim3_seqspacelen_16_4_0pareto : 0.040 seconds
sim3_seqspacelen_16_4_1pareto : 0.039 seconds
sim3_seqspacelen_16_8_0pareto : 0.144 seconds
sim3_seqspacelen_16_8_1pareto : 0.102 seconds
sim3_seqspacelen_16_16_0pareto : 0.375 seconds
sim3_seqspacelen_16_16_1pareto : 0.398 seconds
sim3_seqspacelen_32_2_0pareto : 0.012 seconds
sim3_seqspacelen_32_2_1pareto : 0.044 seconds
sim3_seqspacelen_32_4_0pareto : 0.109 seconds
sim3_seqspacelen_32_4_1pareto : 0.156 seconds
sim3_seqspacelen_32_8_0pareto : 0.290 seconds
sim3_seqspacelen_32_8_1pareto : 0.389 seconds
sim3_seqspacelen_32_16_0pareto : 1.166 seconds
sim3_seqspacelen_32_16_1pareto : 1.821 seconds
sim3_seqspacelen_64_2_0pareto : 0.195 seconds
sim3_seqspacelen_64_2_1pareto : 0.191 seconds
sim3_seqspacelen_64_4_0pareto : 0.567 seconds
sim3_seqspacelen_64_4_1pareto : 0.458 seconds
sim3_seqspacelen_64_8_0pareto : 1.594 seconds
sim3_seqspacelen_64_8_1pareto : 1.679 seconds
sim3_seqspacelen_64_16_0pareto : 7.087 seconds
sim3_seqspacelen_64_16_1pareto : 5.645 seconds
sim3_seqspacelen_128_2_0pareto : 0.713 seconds
sim3_seqspacelen_128_2_1pareto : 0.670 seconds
sim3_seqspacelen_128_4_0pareto : 2.135 seconds
sim3_seqspacelen_128_4_1pareto : 2.418 seconds
sim3_seqspacelen_128_8_0pareto : 6.247 seconds
sim3_seqspacelen_128_8_1pareto : 6.642 seconds
sim3_seqspacelen_128_16_0pareto : 23.922 seconds
sim3_seqspacelen_128_16_1pareto : 24.432 seconds
\end{verbatim}

\subsubsection{Simulation Experiment 10: Vary Feat space }
\label{sim:featspace}
Keeping the seq space at 20 and seq len at 16. vary feat space from 5-10-25-50

\begin{verbatim}
sim3_featspace_5_0pareto : 0.344 seconds
sim3_featspace_5_1pareto : 0.426 seconds
sim3_featspace_5_2pareto : 0.499 seconds
sim3_featspace_10_0pareto : 0.543 seconds
sim3_featspace_10_1pareto : 0.706 seconds
sim3_featspace_10_2pareto : 0.657 seconds
sim3_featspace_25_0pareto : 0.632 seconds
sim3_featspace_25_1pareto : 0.493 seconds
sim3_featspace_25_2pareto : 0.536 seconds
sim3_featspace_50_0pareto : 0.543 seconds
sim3_featspace_50_1pareto : 0.676 seconds
sim3_featspace_50_2pareto : 0.586 seconds
\end{verbatim}

\subsubsection{Simulation Experiment 11: Vary Feat space and sequence length}
\label{sim:featspacelen}
Keeping the seq space at 20, vary feat space from 5-10-25-50, sequence length from seqlen to 64.

\begin{verbatim}
sim3_featspacelen_4_5_0pareto : 0.030 seconds
sim3_featspacelen_4_5_1pareto : 0.037 seconds
sim3_featspacelen_4_10_0pareto : 0.046 seconds
sim3_featspacelen_4_10_1pareto : 0.045 seconds
sim3_featspacelen_4_25_0pareto : 0.056 seconds
sim3_featspacelen_4_25_1pareto : 0.037 seconds
sim3_featspacelen_4_50_0pareto : 0.028 seconds
sim3_featspacelen_4_50_1pareto : 0.071 seconds
sim3_featspacelen_16_5_0pareto : 0.338 seconds
sim3_featspacelen_16_5_1pareto : 0.419 seconds
sim3_featspacelen_16_10_0pareto : 0.536 seconds
sim3_featspacelen_16_10_1pareto : 0.699 seconds
sim3_featspacelen_16_25_0pareto : 0.621 seconds
sim3_featspacelen_16_25_1pareto : 0.515 seconds
sim3_featspacelen_16_50_0pareto : 0.584 seconds
sim3_featspacelen_16_50_1pareto : 0.752 seconds
sim3_featspacelen_32_5_0pareto : 2.014 seconds
sim3_featspacelen_32_5_1pareto : 2.501 seconds
sim3_featspacelen_32_10_0pareto : 2.142 seconds
sim3_featspacelen_32_10_1pareto : 2.176 seconds
sim3_featspacelen_32_25_0pareto : 2.277 seconds
sim3_featspacelen_32_25_1pareto : 2.080 seconds
sim3_featspacelen_32_50_0pareto : 2.555 seconds
sim3_featspacelen_32_50_1pareto : 1.944 seconds
sim3_featspacelen_64_5_0pareto : 8.940 seconds
sim3_featspacelen_64_5_1pareto : 10.489 seconds
sim3_featspacelen_64_10_0pareto : 9.777 seconds
sim3_featspacelen_64_10_1pareto : 8.458 seconds
sim3_featspacelen_64_25_0pareto : 10.321 seconds
sim3_featspacelen_64_25_1pareto : 9.168 seconds
sim3_featspacelen_64_50_0pareto : 9.095 seconds
sim3_featspacelen_64_50_1pareto : 8.893 seconds
\end{verbatim}
%
\subsubsection{Simulation Experiment : K-way multistate design}

\subsubsection{Simulation Experiment 11 : Augmenting Feature space}

\subsection{Benchmarking with other Methods}

\subsection{Experiments on Biological Data}
In this section we apply our algorithm to biological sequence data. These are the following main steps required : 

\begin{itemize}
\item Create alignments of sequences 
\item Train HMM from the alignment
\item Convert HMM representation to CMRF
\item Choose design state and corresponding features
\item Run algorithm to decode sequence
\end{itemize}

\subsubsection{Tied Params - Multistate proteins}



\subsubsection{Profile HMM- Multistate proteins}

\subsubsection{Tied Params - Disordered proteins}







\bibliographystyle{plain}
\bibliography{refs}


\end{document}