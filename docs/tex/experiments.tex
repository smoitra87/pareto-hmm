\section{Experiments}
In this section we evaluate the performance of our algorithm on simulated as well as experimental data. 

\subsection{Simulation Experiments}
We run a variety of different simulation experiments to check the performance of our algorithm on 

\subsubsection{Sim Experiment 1}
\label{sim:toy}
We demonstrate how our algorithm performs on simulated data. The data was prepared by considering a protein of length 12. The sequence space was limited to an alphabet consisting of two arbitrary characters $\alpha,\beta$. The observation belong to a discrete space Helix, Beta and Loop which represented by H,B and L. The weights of the CMRF have weights $\phi,\psi$ as defined in Tables~\ref{tab:psi1} and~\ref{tab:psi2}. We define two states State A and State B that corresponding to competing objectives for multistate design. State A is a Helix-Loop-Helix strand whereas State B is Sheet-Loop-Sheet strand. Figure~\ref{fig:simstates} shows a cartoon diagrams for our design problem. State A and State B have \texttt{HHHHLLLLHHHH} and \texttt{BBBBLLLLBBBB} secondary structure features respectively. These are the observations for the CMRF.
\\
\\
Using the weight matrices we defined in Tables~\ref{tab:psi1} and~\ref{tab:psi2} we run CMRF-Decode to get $X_A$ which satisfies $X_A = \argmin_X E_A(X)$. We obtain $X_B$ similarly. Not surprisingly $X_A$ is $\alpha\alpha\alpha\alpha\alpha\alpha\alpha\alpha\alpha\alpha\alpha\alpha$ 
and $X_B$ is $\beta\beta\beta\beta\beta\beta\beta\beta\beta\beta\beta\beta$. This is because the weights in the table are biased towards $\alpha$ emitting \texttt{H} and $\beta$ emitting \texttt{B}. Thus according to our model $X_A$ is the sequence that is most stable under state A, whereas $X_B$ is the sequence is most stable under state B.
\\
\\
We then run the \texttt{Pareto-Frontier} algorithm using the same weights and observations for state A and state B to enumerate the entire pareto frontier. This pareto frontier contains all the best sequences that any convex combination of $E_A(X)$ and $E_B(X)$ would give rise to. We number the points to show the path taken by the pareto frontier algorithm. We also calculate the energies of all the $2^{12}$ sequences possible under this restricted sequence model and plot the points. The \texttt{Pareto-Frontier} algorithm was able to find the pareto frontier in \textcolor{red}{1 step} steps. See Figure~\ref{fig:pareto-sim} for the trace taken by the algorithm.  We also compare the running times as opposed to brute force and MCMC and compare the running times. 

\begin{table}
\begin{center}
\begin{tabular}{cc|c|c|c|}
\cline{3-5}
& & \multicolumn{3}{c|}{Observations($\mathbf{O_i}$)} \\ \cline{3-5}
& & Helix(H) & Beta(B) & Loop(L) \\ \cline{1-5}
\multicolumn{1}{|c|}{\multirow{2}{*}{Sequence($X_i$)}} &
\multicolumn{1}{|c|}{$\alpha$} & 0.6 & 0.2 & 0.2      \\ \cline{2-5}
\multicolumn{1}{|c|}{}                        &
\multicolumn{1}{|c|}{$\beta$} & 0.2 & 0.6 & 0.2      \\ \cline{1-5}
\end{tabular}
\end{center}
\caption{$\psi(X,\mathbf{O})$ contains the weights for edge potentials of sequence and features together }
\label{tab:psi1}
\end{table}


\begin{table}
\begin{center}
\begin{tabular}{cc|c|c|}
\cline{3-4}
& & \multicolumn{2}{c|}{Sequence($X_{i+1}$)} \\ \cline{3-4}
& & $\alpha$ & $\beta$  \\ \cline{1-4}
\multicolumn{1}{|c|}{\multirow{2}{*}{Sequence($X_i$)}} &
\multicolumn{1}{|c|}{$\alpha$} & 0.7 & 0.3      \\ \cline{2-4}
\multicolumn{1}{|c|}{}                        &
\multicolumn{1}{|c|}{$\beta$} & 0.3 & 0.7      \\ \cline{1-4}
\end{tabular}
\end{center}
\caption{$\psi(X_i,X_{i+1})$ contains the weights for the edge potentials of adjacent amino acids}
\label{tab:psi2}
\end{table}

\begin{table}
\begin{center}
\begin{tabular}{cc|c|c|}
\cline{3-4}
& & \multicolumn{2}{c|}{Sequence($X_{i+1}$)} \\ \cline{3-4}
& & $\alpha$ & $\beta$  \\ \cline{1-4}
\multicolumn{1}{|c|}{\multirow{2}{*}{Sequence($X_i$)}} &
\multicolumn{1}{|c|}{$\alpha$} & 0.7 & 0.3      \\ \cline{2-4}
\multicolumn{1}{|c|}{}                        &
\multicolumn{1}{|c|}{$\beta$} & 0.3 & 0.7      \\ \cline{1-4}
\end{tabular}
\end{center}
\caption{Time and steps taken on simulated data}
\end{table}

\begin{figure}[h!]
    \centering
    \includegraphics[width=0.9\textwidth]{pics/sim1.pdf}
    \caption{Simulation Experiment 1: The emission and transition weights are as defined in Tables~\ref{tab:psi1} and~\ref{tab:psi2}. The Pareto-HMM finds the optimal pareto frontier for a twelve length sequence using the alphabet \texttt\{a,b\} and the feature space \texttt\{H,L,B\} in two steps of the Convex hull algorithm. The magenta stars corresponds to the optimal sequences. The magenta line corresponds to the lower left convex hull of the sequence energy space. The blue stars correspond to non-optimal dominated sequences.  The Pareto-Frontier algoerithm provably finds the frontier quickly instead of having to enumerate,calculate and sort the energies of a  $2^{12}$ sequence space.}
    \label{fig:pareto-sim}
\end{figure}

\pagebreak

\subsubsection{Simulation Experiment 2}
\label{sim:randprobs}
In this example we look at a slightly more complicated energy frontier. To generate this energy surface we resorted to generating random emission and transition probabilities till we found a sufficiently complicated energy frontier but still having alphabet \texttt{\{a,b\}} and feature space \texttt{\{H,L,B\}}. The probabilities for this simulation experiment are shown in tables~\ref{tab:psi3} and~\ref{tab:psi4}. 

\begin{table}
\begin{center}
\begin{tabular}{cc|c|c|c|}
\cline{3-5}
& & \multicolumn{3}{c|}{Observations($\mathbf{O_i}$)} \\ \cline{3-5}
& & Helix(H) & Beta(B) & Loop(L) \\ \cline{1-5}
\multicolumn{1}{|c|}{\multirow{2}{*}{Sequence($X_i$)}} &
\multicolumn{1}{|c|}{$\alpha$} & 0.334 & 0.272 & 0.394      \\ \cline{2-5}
\multicolumn{1}{|c|}{}                        &
\multicolumn{1}{|c|}{$\beta$} & 0.477 & 0.093 & 0.430      \\ \cline{1-5}
\end{tabular}
\end{center}
\caption{$\psi(X,\mathbf{O})$ contains the weights for edge potentials of sequence and features together }
\label{tab:psi3}
\end{table}


\begin{table}
\begin{center}
\begin{tabular}{cc|c|c|}
\cline{3-4}
& & \multicolumn{2}{c|}{Sequence($X_{i+1}$)} \\ \cline{3-4}
& & $\alpha$ & $\beta$  \\ \cline{1-4}
\multicolumn{1}{|c|}{\multirow{2}{*}{Sequence($X_i$)}} &
\multicolumn{1}{|c|}{$\alpha$} & 0.483 & 0.517      \\ \cline{2-4}
\multicolumn{1}{|c|}{}                        &
\multicolumn{1}{|c|}{$\beta$} & 0.589 & 0.411      \\ \cline{1-4}
\end{tabular}
\end{center}
\caption{$\psi(X_i,X_{i+1})$ contains the weights for the edge potentials of adjacent amino acids}
\label{tab:psi4}
\end{table}


\begin{figure}[h!]
    \centering
    \includegraphics[width=0.9\textwidth]{pics/sim2_all.pdf}
    \caption{Simulation Experiment 2: Pareto Frontier algorithm is applied to a more complicated energy surface. The algorithm finds the pareto frontier denoted by the dashed magenta in 5 steps, which is significantly faster than brute force enumeration.  }
    \label{fig:sim2_all}
\end{figure}

\begin{figure}[h!]
    \centering
    \includegraphics[width=0.9\textwidth]{pics/sim2_frontier.pdf}
    \caption{Simulation Experiment 2(Frontier): This shows the results from simulation 2. The sequences along the frontier are shown as well }
    \label{fig:sim2_frontier}
\end{figure}

\pagebreak
\clearpage

\subsubsection{Simulation Experiment 3: Random probs tied}
\label{sim:randprobstied}
In this experiment we use a 12 length sequence. We do 2-state design using features \texttt{HHHHLLLLHHHH} and \texttt{BBBBLLLLBBBB} as before. The sequence space is restricted to a dummy \texttt{\{a,b\}}  We randomize the emission and transition probabilities. The parameters of the emission and the transition are ties together.  We store the frontier, frontier energies, computation time taken and 

\subsubsection{Simulation Experiment 4:Random probs untied}
\label{sim:randprobsuntied}
Same as Simulation~\ref{sim:randprobstied} with the difference that the emission and the transition probabilities are untied. 

\subsubsection{Simulation Experiment 5 : Random feature space tied}
Like~\ref{sim:randprobstied} but the feature space is random
\label{sim:randfeatstied}

\subsubsection{Simulation Experiment 6 :  Random feature space untied}
Like~\ref{sim:randprobsuntied} but the feature space is random
\label{sim:randfeatsuntied}

\subsubsection{Simulation Experiment 7: Vary Seq space }

\subsubsection{Simulation Experiment 8: Seq space 20, Vary seq length}

\subsubsection{Simulation Experiment 9 : Jointly vary seq space and length}

\subsubsection{Simulation Experiment 10 : K-way multistate design}

\subsection{Benchmarking with other Methods}

\subsection{Experiments on Biological Data}


